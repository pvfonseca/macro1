\documentclass[10pt]{beamer}
\usetheme{jambro}

\title[]{Curva de Phillips, taxa natural de desemprego e inflação}
\author[]{Paulo Victor da Fonseca}
\date{11 de maio de 2023}

\hypersetup{
    colorlinks = true,
    urlcolor = teal,
    linkcolor = teal    
}
\usepackage[portuguese]{babel}
\usepackage{subfig}
\usepackage{emoji}
\usepackage{hyperref}

\begin{document}

\begin{frame}[plain]
    \titlepage{
        \begin{center}
            \begin{minipage}{0.8\textwidth}
                \centering
            \end{minipage}
        \end{center}}
\end{frame}

\begin{frame}{Sumário}
    \tableofcontents
\end{frame}

\section{Curva de Phillips e taxa natural de desemprego}
\begin{frame}
    {Curva de Phillips e taxa natural de desemprego}
    \begin{itemize}
        \item Curva de Phillips (PC) intimamente com taxa natural de desemprego\bigskip
        \item Implicação da PC original: inexistência de algo como uma taxa natural de desemprego para o qual a economia convergiria no médio prazo\bigskip
        \item Se formuladores de política estivessem dispostos a tolerar uma taxa de inflação mais alta, poderiam manter uma taxa de desemprego mais baixa de forma permanente\bigskip
        \item Trade-off que parecia ser corroborado pela evidência empírica da década de 1960
    \end{itemize}
\end{frame}

\section{Bibliografia}
\begin{frame}{\emoji{books} Bibliografia}
    \begin{itemize}                
        \item BLANCHARD, O. Macroeconomia. 7.ed. São Paulo: Pearson Education do Brasil, 2017\medskip                
        \item CARLIN, W.; SOSKICE, D. Macroeconomics: Institutions, instability, and the financial system. Oxford, UK: Oxford University Press, 2015\medskip        
    \end{itemize}
\end{frame}
\end{document}