\documentclass[10pt]{beamer}
\usetheme{jambro}

\title[]{Mercado de Trabalho}
\author[]{Paulo Victor da Fonseca}
\date{}

\hypersetup{
    colorlinks = true,
    urlcolor = teal,
    linkcolor = teal    
}
\usepackage[portuguese]{babel}
\usepackage{subfig}
\usepackage{emoji}
\usepackage{hyperref}

\begin{document}

\begin{frame}[plain]
    \titlepage{
        \begin{center}
            \begin{minipage}{0.8\textwidth}
                \centering
            \end{minipage}
        \end{center}}
\end{frame}

\begin{frame}{Sumário}
    \tableofcontents
\end{frame}

\section{Introdução}
\begin{frame}
    {Introdução}
    \begin{itemize}
        \item Até agora: preços constantes $\Rightarrow$ firmas capazes e dispostas a ofertar qualquer montante de produto a um dado nível de preços\bigskip
        \item Hipótese aceitável para \textcolor{purple}{curto prazo}\bigskip
        \item \hlight{Médio prazo}: hipótese de $\bar{P}$ deve ser abandonada - explorar ajustes de preços e salários ao longo do tempo e determinar como isso afeta produto agregado\bigskip
        \item Formularemos um modelo para \hlight{mercado de trabalho}, no qual salários são determinados
    \end{itemize}
\end{frame}

\begin{frame}
    {Introdução}
    \begin{figure}
        \centering
        \href{https://www.core-econ.org/the-economy/book/text/09.html}{\includegraphics[width=0.65\textwidth]{./figures/aula10_fig1.PNG}}
        \caption{Salário real $\times$ preço do minério de ferro - Western Australia. Fonte: \href{https://www.core-econ.org/the-economy/book/text/09.html}{CORE-Econ}}
    \end{figure}
\end{frame}


\begin{frame}
    {Introdução}
    \begin{figure}
        \centering
        \href{https://www.core-econ.org/the-economy/book/text/09.html}{\includegraphics[width=0.6\textwidth]{./figures/aula10_fig2.PNG}}
        \caption{Salário real $\times$ taxa de desemprego - Western Australia. Fonte: \href{https://www.core-econ.org/the-economy/book/text/09.html}{CORE-Econ}}
    \end{figure}
\end{frame}

\section{Mercado de Trabalho}
\begin{frame}{Mercado de Trabalho}
    \begin{itemize}
        \item Começaremos a descrever funcionamento dos mercados de trabalho\bigskip
        \item Por que, mesmo em equilíbrio, a oferta de trabalho excede a demanda por trabalho?\bigskip
        \item Nessas condições: \hlight{desemprego involuntário} ($\times$ desemprego voluntário/friccional)\bigskip
        \item \href{http://www.ilo.org/}{Organização Internacional do Trabalho (OIT)} - definição padronizada de desempregado como pessoas que:\medskip
        \begin{enumerate}
            \item estavam sem trabalho durante um período de referência (usualmente 4 semanas) - não estavam em emprego com carteira assinada ou como autônomo\medskip
            \item estavam disponíveis para trabalhar\medskip
            \item estavam procurando emprego - ativamente tomaram medidas naquele período para encontrar trabalho (carteira assinada ou autônomo)
        \end{enumerate}
    \end{itemize}
\end{frame}

\begin{frame}
    {Mercado de Trabalho}
    \begin{figure}
        \centering
        \href{https://www.core-econ.org/the-economy/book/text/09.html}{\includegraphics[width=\textwidth]{./figures/aula10_fig3.PNG}}
        \caption{Mercado de Trabalho. Fonte: \href{https://www.core-econ.org/the-economy/book/text/09.html}{CORE-Econ}}
    \end{figure}
\end{frame}

\begin{frame}{Mercado de Trabalho}
    \begin{center}
		\begin{minipage}[b]{.55\textwidth}
			\begin{tikzpicture}
				\node[inner sep=0, align=left] (image) at (0,0) {\includegraphics[width=\textwidth]{./figures/aula10_fig4.PNG}};					
			\end{tikzpicture}
			\tiny{{\scshape FIGURA}. \ Mercado de Trabalho - EUA 2014 (em milhões). Fonte: Blanchard (2017)} 
		\end{minipage}
	\end{center}
\end{frame}

\begin{frame}{Mercado de Trabalho}
    \begin{center}
		\begin{minipage}[b]{.55\textwidth}
			\begin{tikzpicture}
				\node[inner sep=0, align=left] (image) at (0,0) {\includegraphics[width=\textwidth]{./figures/aula10_fig4.PNG}};
				\draw[arrow,->,opacity=1] (2.5,1.3) to[bend left] +(+1,+0) node[anchor=west,opacity=1] {\normalsize{\hand \begin{tabular}{l}
    Exclui: \\
    - < 16 anos \\
    - alistamentos F.A. \\ 
    - população encarcerada
\end{tabular}}};				
			\end{tikzpicture}
			\tiny{{\scshape FIGURA}. \ Mercado de Trabalho - EUA 2014 (em milhões). Fonte: Blanchard (2017)} 
		\end{minipage}
	\end{center}
\end{frame}

\begin{frame}{Mercado de Trabalho}
    \begin{center}
		\begin{minipage}[b]{.55\textwidth}
			\begin{tikzpicture}
				\node[inner sep=0, align=right] (image) at (0,0) {\includegraphics[width=\textwidth]{./figures/aula10_fig4.PNG}};
				\draw[arrow,->,opacity=1] (-1.5,0.3) to[bend right] +(-1,+0) node[anchor=east,opacity=1] {\normalsize{\hand \begin{tabular}{l}
    Taxa de participação: \\
    - F.T./P.I.A. \\
    - 63\%
\end{tabular}}};				
			\end{tikzpicture}
			\tiny{{\scshape FIGURA}. \ Mercado de Trabalho - EUA 2014 (em milhões). Fonte: Blanchard (2017)} 
		\end{minipage}
	\end{center}
\end{frame}

\begin{frame}{Mercado de Trabalho}
    \begin{center}
		\begin{minipage}[b]{.55\textwidth}
			\begin{tikzpicture}
				\node[inner sep=0, align=left] (image) at (0,0) {\includegraphics[width=\textwidth]{./figures/aula10_fig4.PNG}};
				\draw[arrow,->,opacity=1] (0.7,-2) to[bend left] +(+1.2,-0.5) node[anchor=west,opacity=1] {\normalsize{\hand \begin{tabular}{l}
    Taxa de desemprego: \\
    - desempregados/F.T. \\
    - 6,1\%
\end{tabular}}};				
			\end{tikzpicture}
			\tiny{{\scshape FIGURA}. \ Mercado de Trabalho - EUA 2014 (em milhões). Fonte: Blanchard (2017)} 
		\end{minipage}
	\end{center}
\end{frame}

\begin{frame}{Mercado de Trabalho}
    \begin{center}
		\begin{minipage}[b]{.55\textwidth}
			\begin{tikzpicture}
				\node[inner sep=0, align=right] (image) at (0,0) {\includegraphics[width=\textwidth]{./figures/aula10_fig5.PNG}};
				\draw[arrow,->,opacity=1,blue] (-2.1,0.4) to[bend right] +(-1,-0.2) node[anchor=east,opacity=1] {\normalsize{\hand \begin{tabular}{l}
    Taxa de participação: \\    
    - 73,9\%
\end{tabular}}};				
			\end{tikzpicture}
			\tiny{{\scshape FIGURA}. \ Mercado de Trabalho - União Europeia 2014 (em milhões). Fonte: Blanchard, Amighini, Giavazzi (2017)} 
		\end{minipage}
	\end{center}
\end{frame}

\begin{frame}{Mercado de Trabalho}
    \begin{center}
		\begin{minipage}[b]{.55\textwidth}
			\begin{tikzpicture}
				\node[inner sep=0, align=left] (image) at (0,0) {\includegraphics[width=\textwidth]{./figures/aula10_fig5.PNG}};
				\draw[arrow,->,opacity=1,blue] (-0.2,-1.5) to[bend left] +(+1,-0.5) node[anchor=west,opacity=1] {\normalsize{\hand \begin{tabular}{l}
    Taxa de desemprego: \\    
    - 10,2\%
\end{tabular}}};				
			\end{tikzpicture}
			\tiny{{\scshape FIGURA}. \ Mercado de Trabalho - União Europeia 2014 (em milhões). Fonte: Blanchard, Amighini, Giavazzi (2017)} 
		\end{minipage}
	\end{center}
\end{frame}

\begin{frame}{Mercado de Trabalho}
    \begin{center}
		\begin{minipage}[b]{.55\textwidth}
			\begin{tikzpicture}
				\node[inner sep=0, align=left] (image) at (0,0) {\href{https://www.ibge.gov.br/explica/desemprego.php}{\includegraphics[width=\textwidth]{./figures/aula10_fig6.PNG}}};								
			\end{tikzpicture}
			\tiny{{\scshape FIGURA}. \ Mercado de Trabalho - Classificação IBGE. Fonte: \href{https://www.ibge.gov.br/explica/desemprego.php}{IBGE}} 
		\end{minipage}
	\end{center}
\end{frame}

\begin{frame}{Mercado de Trabalho}
    \begin{center}
		\begin{minipage}[b]{.9\textwidth}
			\begin{tikzpicture}
				\node[inner sep=0, align=left] (image) at (0,0) {\href{https://www.ibge.gov.br/explica/desemprego.php}{\includegraphics[width=\textwidth]{./figures/aula10_fig7.PNG}}};								
			\end{tikzpicture}
			\tiny{{\scshape FIGURA}. \ Mercado de Trabalho - Brasil (2022.4). Fonte: \href{https://www.ibge.gov.br/explica/desemprego.php}{IBGE}} 
		\end{minipage}
	\end{center}
\end{frame}

\begin{frame}{Mercado de Trabalho}
    \begin{itemize}
        \item Uma dada taxa de desemprego pode refletir realidades bem diversas:\bigskip
        \begin{enumerate}
            \item Mercado de trabalho ativo: muitos \textcolor{purple}{desligamentos} e muitas \textcolor{blue}{admissões}\medskip
            \item Mercado de trabalho \hlight{esclerosado}: poucos desligamentos, poucas admissões e contingente estagnado de desempregados
        \end{enumerate}
    \end{itemize}
\end{frame}

\begin{frame}{Mercado de Trabalho}
\begin{itemize}
		\item \href{https://www.census.gov/programs-surveys/cps.html}{Current Population Survey (CPS) - US Census Bureau}
	\end{itemize}
    \begin{center}
		\begin{minipage}[b]{.45\textwidth}
			\begin{tikzpicture}
				\node[inner sep=0, align=left] (image) at (0,0) {\includegraphics[width=\textwidth]{./figures/aula10_fig8.PNG}};
				\draw[arrow,->,opacity=1] (1.3,1.3) to[bend left] +(+1,+0) node[anchor=west,opacity=1] {\normalsize{\hand \begin{tabular}{l}
    Grandes fluxos: \\
    - 8,2 milhões desligamentos/mês \\
    - 75\% demissões voluntárias \\ 
    - 25\% demissões involuntárias
\end{tabular}}};				
			\end{tikzpicture}
			\tiny{{\scshape FIGURA}. \ Fluxos médios mensais - EUA 1996-2014 (em milhões). Fonte: Blanchard (2017)} 
		\end{minipage}
	\end{center}
\end{frame}

\begin{frame}{Mercado de Trabalho}
\begin{itemize}
		\item \href{https://www.census.gov/programs-surveys/cps.html}{Current Population Survey (CPS) - US Census Bureau}
	\end{itemize}
    \begin{center}
		\begin{minipage}[b]{.45\textwidth}
			\begin{tikzpicture}
				\node[inner sep=0, align=left] (image) at (0,0) {\includegraphics[width=\textwidth]{./figures/aula10_fig8.PNG}};
				\draw[arrow,->,opacity=1, brick] (1.3,1.3) to[bend left] +(+1,+0) node[anchor=west,opacity=1] {\normalsize{\hand \begin{tabular}{l}
    Relação ao \# de desempregados: \\
    - 44\% dos desempregados deixam \\
    essa condição a cada mês \\
    - Duração média desemprego: \\
    2-3 meses
\end{tabular}}};				
			\end{tikzpicture}
			\tiny{{\scshape FIGURA}. \ Fluxos médios mensais - EUA 1996-2014 (em milhões). Fonte: Blanchard (2017)} 
		\end{minipage}
	\end{center}
\end{frame}

\begin{frame}{Mercado de Trabalho}
\begin{itemize}
		\item \href{https://www.census.gov/programs-surveys/cps.html}{Current Population Survey (CPS) - US Census Bureau}
	\end{itemize}
    \begin{center}
		\begin{minipage}[b]{.45\textwidth}
			\begin{tikzpicture}
				\node[inner sep=0, align=left] (image) at (0,0) {\includegraphics[width=\textwidth]{./figures/aula10_fig8.PNG}};
				\draw[arrow,->,opacity=1, blue] (1.3,1.3) to[bend left] +(+1,+0) node[anchor=west,opacity=1] {\normalsize{\hand \begin{tabular}{l}
    Fluxos elevados in-out F.T.: \\
    \emoji{warning} Trabalhadores desalentados \\    
    - Foco na taxa de emprego
\end{tabular}}};				
			\end{tikzpicture}
			\tiny{{\scshape FIGURA}. \ Fluxos médios mensais - EUA 1996-2014 (em milhões). Fonte: Blanchard (2017)} 
		\end{minipage}
	\end{center}
\end{frame}

\begin{frame}{Mercado de Trabalho}
    \begin{footnotesize}
        \begin{table}[h!]
            \centering
            \begin{tabular}{ |l||c|c|c|c|c|c|c|c|  }
                \hline
                \multicolumn{9}{|c|}{Duração média do desemprego (meses)}                \\
                \hline
                \hline
                País             & 2014 & 2015 & 2016 & 2017 & 2018 & 2019 & 2020 & 2021 \\
                \hline
                \rowcolor{gray!10} Canadá           & 4,3  & 4,1  & 4,1  & 4,1  & 3,8  & 3,5  & 3,3  & 4,9  \\
                Colômbia         & 4,3  & 4,0  & 4,1  & 4,3  & 4,7  & 5,0  & 4,7  & 6,5  \\
                \rowcolor{gray!10} Eslováquia       & 29,6 & 31,5 & 29,1 & 30,2 & 29,8 & 28,5 & 23,8 & 20,6 \\
                Estados Unidos   & 7,8  & 6,7  & 6,3  & 5,8  & 5,2  & 5,0  & 3,8  & 6,6  \\
                \rowcolor{gray!10} Finlândia        & 10,7 & 10,5 & 10,4 & 10,7 & 10,2 & 9,7  & 8,2  & 11,3 \\
                França           & 14,2 & 14,6 & 15,6 & 15,5 & 14,5 & 13,9 & 13,4 & -    \\
                \rowcolor{gray!10} Hungria          & 18,2 & 18,1 & 18,1 & 16,0 & 15,2 & 12,4 & 9,9  & 8,4  \\
                República Tcheca & 17,6 & 19,4 & 18,3 & 15,6 & 14,0 & 13,8 & 10,4 & 11,8 \\
                \rowcolor{gray!10} Suíça            & 16,9 & 17,2 & 17,3 & 17,0 & 18,0 & 17,6 & 16,0 & 17,8 \\
                \hline
            \end{tabular}\vspace{.2cm}\newline
			\tiny{{\scshape TABELA.} Duração média do desemprego em meses. Fonte: \href{https://stats.oecd.org/index.aspx?DataSetCode=AVD_DUR}{OECD.stats}\newline}%
        \end{table}
    \end{footnotesize}
\end{frame}

\begin{frame}{Mercado de Trabalho}
    \begin{center}
		\begin{minipage}[b]{.9\textwidth}
			\begin{tikzpicture}
				\node[inner sep=0, align=left] (image) at (0,0) {\href{https://data.oecd.org/chart/6Rq2}{\includegraphics[width=\textwidth]{./figures/aula10_fig9.PNG}}};
			\end{tikzpicture}
			\tiny{{\scshape FIGURA}. \ Tendência e heterogeneidade no desemprego (seleção OCDE): 1955-2022. Fonte: \href{https://data.oecd.org/chart/6Rq2}{OCDE}} 
		\end{minipage}
	\end{center}
\end{frame}

\begin{frame}{Mercado de Trabalho}
    \begin{itemize}
        \item Taxa de desemprego varia ao longo do tempo e entre países\bigskip
        \item Elevado grau de heterogeneidade\bigskip
        \item Pós CFG2008 (entre 2009 e 2012):\medskip
            \begin{itemize}
                \item Holanda: desemprego médio de 4,5\%\medskip
                \item Espanha: desemprego médio de 21,2\%\bigskip
            \end{itemize}
        \item Mesmo país pode apresentar grandes variações ao longo do tempo\bigskip
        \item Irlanda:\medskip
        \begin{itemize}
            \item Desemprego $\approx$ 16\% final dos anos 1980s\medskip
            \item Começo dos anos 2000s, desemprego de $\approx$ 4\%\medskip
            \item Início da crise, aumentando novamente para 14\%
        \end{itemize}
    \end{itemize}
\end{frame}

\section{Dinâmica do Desemprego}
\begin{frame}{Dinâmica do Desemprego}
    \begin{center}
		\begin{minipage}[b]{\textwidth}
			\begin{tikzpicture}
				\node[inner sep=0, align=left] (image) at (0,0) {\includegraphics[width=\textwidth]{./figures/aula10_fig10.PNG}};
			\end{tikzpicture}
			\tiny{{\scshape FIGURA}. \ Dinâmica da taxa de desemprego, EUA (1948-2014). Fonte: Blanchard (2017)} 
		\end{minipage}
	\end{center}
\end{frame}

\begin{frame}{Dinâmica do Desemprego}
    \begin{itemize}
        \item Até meados dos 1980s: tendência de alta da taxa de desemprego\medskip
        \begin{itemize}
            \item Década de 1950: média de 4,5\%\medskip
            \item Década de 1960: média de 4,7\%\medskip
            \item Década de 1970: média de 6,2\%\medskip
            \item Década de 1980: média de 7,3\%\medskip
        \end{itemize}
        \item Desde então, taxa de desemprego declinou continuamente por mais de 2 décadas\bigskip
        \item CFG2008: taxa aumentou acentuadamente, mas tornou a baixar de novo
    \end{itemize}
\end{frame}

\begin{frame}{Dinâmica do Desemprego}
    \begin{itemize}
        \item Deixando de lado oscilações de tendência, \hlight{os movimentos anuais da taxa de desemprego estão fortemente associados a recessões e expansões}\bigskip
        \item Desemprego é custoso pois representa subutilização de recursos e, além disso, está associado a infelicidade e estresse psicológico\bigskip
        \item Como flutuações da taxa de desemprego afetam trabalhadores individualmente?\medskip
        \begin{enumerate}
            \item Efeito dos movimentos da taxa de desemprego agregado sobre bem-estar individual dos trabalhadores\medskip
            \item Efeito da taxa de desemprego agregado sobre salários
        \end{enumerate}
    \end{itemize}
\end{frame}

\begin{frame}{Dinâmica do Desemprego}
    \begin{itemize}
        \item Como empresas podem reduzir emprego frente a redução de demanda?\bigskip
        \item Podem frear admissão de novos funcionários ou demitir empregados\bigskip
        \item Normalmente, a primeira opção é preferível (contando com demissões voluntárias ou aposentadorias)\bigskip
        \item Mas pode não ser suficiente se a redução na demanda for grande e, então, empresas podem ter de demitir funcionários
    \end{itemize}
\end{frame}

\begin{frame}{Dinâmica do Desemprego}
    \begin{itemize}
        \item Implicações para empregados e desempregados:\medskip
        \begin{enumerate}
            \item Ajuste via número menor de admissões: probabilidade de desempregado encontrar emprego diminuirá - menos admissões $\Rightarrow$ menor abertura de postos de trabalho e maior desemprego $\Rightarrow$ mais candidatos para postos de trabalho\medskip
            \item Ajuste via demissões involuntárias: empregados terão risco maior de perder seus empregos\bigskip
        \end{enumerate}
        \item De modo geral, empresas recorrem às duas formas de ajuste - desemprego maior relacionado tanto com probabilidade menor de desempregados encontrarem emprego quanto probabilidade maior de empregados perderem o emprego
    \end{itemize}
\end{frame}

\begin{frame}{Dinâmica do Desemprego}
    \begin{center}
		\begin{minipage}[b]{.9\textwidth}
			\begin{tikzpicture}
				\node[inner sep=0] (image) at (0,0) {\includegraphics[width=\textwidth]{./figures/aula10_fig11.PNG}};
			\end{tikzpicture}
			\tiny{{\scshape FIGURA}. \ Taxa de desemprego e proporção de desempregados que encontram emprego, EUA (1996-2014). Fonte: Blanchard (2017)} 
		\end{minipage}
	\end{center}
\end{frame}

\begin{frame}{Dinâmica do Desemprego}
    \begin{center}
		\begin{minipage}[b]{.9\textwidth}
			\begin{tikzpicture}
				\node[inner sep=0] (image) at (0,0) {\includegraphics[width=\textwidth]{./figures/aula10_fig12.PNG}};
			\end{tikzpicture}
			\tiny{{\scshape FIGURA}. \ Taxa de desemprego e taxa mensal de desligamento, EUA (1996-2014). Fonte: Blanchard (2017)} 
		\end{minipage}
	\end{center}
\end{frame}

\begin{frame}{Dinâmica do Desemprego}
    \begin{itemize}
        \item Em resumo, se desemprego é alto, a situação dos trabalhadores piora em dois aspectos:\bigskip
        \begin{enumerate}
            \item Empregados se defrontam com maior probabilidade de perder emprego\medskip
            \item Desempregados se defrontam com probabilidade mais baixa de encontrar emprego, i.e., maior probabilidade de permanecerem desempregados por período mais longo
        \end{enumerate}
    \end{itemize}
\end{frame}

\begin{frame}{Dinâmica do Desemprego}
    \begin{center}
		\begin{minipage}[b]{.8\textwidth}
			\begin{tikzpicture}
				\node[inner sep=0] (image) at (0,0) {\href{https://www.ibge.gov.br/estatisticas/sociais/trabalho/9173-pesquisa-nacional-por-amostra-de-domicilios-continua-trimestral.html?=&t=series-historicas&utm_source=landing&utm_medium=explica&utm_campaign=desemprego}{\includegraphics[width=\textwidth]{./figures/aula10_fig13.PNG}}};
			\end{tikzpicture}
			\tiny{{\scshape FIGURA}. \ Taxa de desocupação, Brasil (2012-2023). Fonte: \href{https://www.ibge.gov.br/estatisticas/sociais/trabalho/9173-pesquisa-nacional-por-amostra-de-domicilios-continua-trimestral.html?=&t=series-historicas&utm_source=landing&utm_medium=explica&utm_campaign=desemprego}{PNAD contínua - IBGE}} 
		\end{minipage}
	\end{center}
\end{frame}

\section{Determinação de Salários}
\begin{frame}
    {Determinação de salários: Introdução}
    \begin{itemize}
        \item Foco agora: determinação de salários e relação entre salários e desemprego\bigskip
        \item Salários podem ser fixados de várias formas\bigskip
        \item Podem ser determinados por \hlight{negociação coletiva}: negociação entre firmas e sindicatos\bigskip
        \item EUA: negociação coletiva tem papel limitado - $< 10\%$ dos trabalhadores com salários fixados por acordos coletivos\bigskip
        \item Para o restante dos trabalhadores, salários fixados ou por empregadores ou por negociações individuais\bigskip
        \item Quanto maior a qualificação necessária para o emprego, maior a probabilidade de haver negociação
    \end{itemize}
\end{frame}

\begin{frame}
    {Determinação de salários: Introdução}
    \begin{itemize}
        \item Diferenças entre países: e.g., negociação coletiva tem papel importante no Japão e maioria dos países europeus\bigskip
        \item Podem ser realizadas no nível das firmas, nível setorial ou nível nacional\bigskip
        \item Às vezes, acordos feitos por contrato aplicam-se apenas às empresas que assinaram o acordo\bigskip
        \item Outras, são estendidos automaticamente a todas as firmas e trabalhadores do setor ou da economia
    \end{itemize}
\end{frame}

\begin{frame}{Determinação de salários: Introdução}
    \begin{center}
		\begin{minipage}[b]{.55\textwidth}
			\begin{tikzpicture}
				\node[inner sep=0] (image) at (0,0) {\href{https://ilostat.ilo.org/topics/union-membership/}{\includegraphics[width=\textwidth]{./figures/aula10_fig14.PNG}}};
			\end{tikzpicture}
			\tiny{{\scshape FIGURA}. \ Taxa de densidade de sindicatos. Fonte: \href{https://ilostat.ilo.org/topics/union-membership/}{ILOSTAT}} 
		\end{minipage}
	\end{center}
\end{frame}

\begin{frame}{Determinação de salários: Introdução}
    \begin{center}
		\begin{minipage}[b]{.6\textwidth}
			\begin{tikzpicture}
				\node[inner sep=0] (image) at (0,0) {\href{https://ilostat.ilo.org/topics/union-membership/}{\includegraphics[width=\textwidth]{./figures/aula10_fig15.PNG}}};
			\end{tikzpicture}
			\tiny{{\scshape FIGURA}. \ Taxa de densidade de sindicatos. Fonte: \href{https://ilostat.ilo.org/topics/union-membership/}{ILOSTAT}} 
		\end{minipage}
	\end{center}
\end{frame}

\begin{frame}
    {Determinação de salários: Introdução}
    \emoji{question} Dadas diferenças entre trabalhadores e países, é possível formular uma teoria ``geral'' de determinação de salários\bigskip
    \begin{itemize}        
        \item Embora diferenças institucionais influenciem fixação de salários, há forças comuns em ação em todos os países\bigskip
        \item Dois conjuntos de fatores mais importantes:\medskip
        \begin{enumerate}
            \item Trabalhadores recebem salário que excede \textcolor{purple}{salário de reserva} (nível de salário que torna trabalhadores indiferentes entre trabalhar e permanecer desempregados)\medskip
            \item Salários normalmente dependem das condições do mercado de trabalho: quanto menor a taxa de desemprego, maiores os salários
        \end{enumerate}
    \end{itemize}
\end{frame}

\begin{frame}{Determinação de salários: Introdução}
    \begin{center}
		\begin{minipage}[b]{.8\textwidth}
			\begin{tikzpicture}
				\node[inner sep=0] (image) at (0,0) {\includegraphics[width=\textwidth]{./figures/aula10_fig16.PNG}};
			\end{tikzpicture}
			\tiny{{\scshape FIGURA}. \ Curva de determinação de salários. Fonte: Carlin e Soskice (2015)} 
		\end{minipage}
	\end{center}
\end{frame}

\begin{frame}
    {Determinação de salários: introdução}
    \begin{itemize}
        \item Duas grandes linhas de raciocínio:\medskip
        \begin{enumerate}
            \item Mesmo na ausência de negociações coletivas, trabalhadores ainda tem algum poder de negociação que usam para obter uma remuneração superior ao salário de reserva\medskip
            \item Empresas podem, por vários motivos, desejar pagar remunerações mais altas que o salário de reserva
        \end{enumerate}
    \end{itemize}
\end{frame}

\begin{frame}
    {Negociação}
    \begin{itemize}
        \item \textcolor{purple}{Poder de negociação} do trabalhador depende, essencialmente, de dois fatores:\medskip
        \begin{enumerate}
            \item Custo para firma para substituí-lo caso deixe a empresa\medskip
            \item Dificuldade que teria para encontrar um emprego caso deixasse a empresa\bigskip
        \end{enumerate}
        \item Em termos gerais, quanto maior o custo da empresa em substituir um trabalhador e quanto mais fácil para ele encontrar um emprego, maior o poder de negociação\bigskip
        \item Duas implicações:\medskip
        \begin{enumerate}
            \item Poder de negociação depende, em primeiro lugar, da natureza do trabalho: diferenças de qualificação\medskip
            \item Poder de negociação depende das condições do mercado de trabalho. Baixa taxa de desemprego $\Rightarrow$ mais difícil para firma encontrar substituto e, ao mesmo tempo, mais fácil para trabalhador encontrar outro emprego
        \end{enumerate}
    \end{itemize}
\end{frame}

\begin{frame}
    {Salário-eficiência}
    \begin{itemize}
        \item Independente do poder de negociação dos trabalhadores, firmas podem desejar pagar salário acima que o de reserva\bigskip
        \item Salários mais altos podem fazer com que trabalhadores sejam mais produtivos\bigskip
        \item Se leva tempo para que funcionários aprendam como realizar um trabalho corretamente, firmas desejarão que permaneçam por mais tempo\bigskip
        \item Com salário no nível de reserva, funcionários são indiferentes entre permanecer empregados ou sair\bigskip
        \item Neste caso, muitos pediriam demissão voluntária e rotatividade seria elevada\bigskip
        \item Salário fixado acima do de reserva torna permanência de funcionários financeiramente atrativa - diminui rotatividade e aumenta produtividade
    \end{itemize}
\end{frame}

\begin{frame}
    {Salário-eficiência}
    \begin{itemize}
        \item \hlight{Teorias de salário-eficiência}: relacionam produtividade ou eficiência dos trabalhadores ao salário que recebem\bigskip
        \item Assim como teorias baseadas na negociação, sugerem que salários dependem tanto da natureza do emprego quanto das condições do mercado de trabalho:\medskip
        \begin{enumerate}
            \item Empresas como de alta tecnologia pagam mais que as de setores nos quais as atividades são mais rotineiras\medskip
            \item Taxa de desemprego baixa torna demissão voluntária mais atrativa para funcionários - mais fácil encontrar novos empregos. Ou seja, quando desemprego diminui, empresas que desejam evitar aumento de demissões voluntárias aumentam salários para induzir trabalhadores a permanecerem. Portanto, desemprego baixo leva a salários mais altos
        \end{enumerate}
    \end{itemize}
\end{frame}

\begin{frame}
    {Salários, preços e desemprego}
    \begin{itemize}
        \item A discussão anterior sobre determinação de salários sugere a seguinte equação:
        \begin{equation}
            W = P^eF(u,z), \qquad F_u(\bullet) < 0, F_z(\bullet) > 0
        \end{equation}
        \item Salário nominal agregado, $W$, depende de três fatores:
        \begin{enumerate}
            \item Nível esperado de preços, $P^e$\medskip
            \item Taxa de desemprego, $u$\medskip
            \item Variável abrangente, $z$, que representa todas as outras variáveis que podem afetar o resultado da fixação dos salários
        \end{enumerate}
    \end{itemize}
\end{frame}

\begin{frame}
    {Nível esperado de preços}
    \begin{itemize}
        \item Ignorando, por enquanto, diferença entre nível de preços esperado e efetivo, por que nível de preços afeta salários?\bigskip
        \item Tanto trabalhadores quanto firmas preocupam-se com \textcolor{purple}{salários reais}, e não nominais\bigskip
        \item Se trabalhadores esperam que nível de preços irá dobrar, solicitarão que salário nominal dobre\bigskip
        \item Se empresas esperam que nível de preços dobre, estarão dispostos a dobrar o salário nominal\bigskip
        \item Portanto, ambos concordarão em dobrar o salário nominal, mantendo o salário real constante
    \end{itemize}
\end{frame}

\begin{frame}
    {Nível esperado de preços}
    \begin{itemize}
        \item Por que nível \hlight{esperado} de preços, $P^e$, ao invés do nível efetivo, $P$?\bigskip
        \item Salários são fixados em termos nominais e, no momento em que são fixados, ainda não se conhece o nível de preços relevante\bigskip
        \item Por exemplo: alguns dos contratos de sindicatos nos EUA fixam salários nominais antecipadamente por 3 anos. Portanto, sindicatos e empresas devem decidir quais serão os salários nominais nos três anos seguintes com base no que esperam que aconteça com o nível de preços\bigskip
        \item Mesmo com salários fixados por empresas ou negociações individuais, normalmente, salários nominais são fixados por um ano. Se nível de preços sobe de modo inesperado durante este período, salários nominais normalmente não serão reajustados\bigskip
        \item Portanto, é fundamental analisarmos, também, como agentes formam expectativas sobre nível de preços - Macro III
    \end{itemize}
\end{frame}

\begin{frame}
    {Taxa de desemprego}
    \begin{itemize}
        \item Sinal algébrico da primeira derivada da função $F(u,z)$ com relação a $u$ indica que aumentos da taxa de desemprego diminuem os salários\bigskip
        \item Se salários são determinados por negociação, um desemprego mais alto enfrequece o poder de barganha dos trabalhadores, forçando-os a aceitar salários mais baixos\bigskip
        \item Se salários são determinados por considerações de salário-eficiência, um desemprego mais alto permite que empresas paguem salários mais baixos e, ainda assim, manteham funcionários dispostos a trabalhar
    \end{itemize}
\end{frame}

\begin{frame}
    {Outros fatores}
    \begin{itemize}
        \item A variável abrangente, $z$, representa fatores que afetam os salários, dados $P^e$ e $u$\bigskip
        \item Por convenção, definiremos $z$ de modo que um aumento dessa variável leve a um aumento do salário nominal agregado\bigskip
        \begin{enumerate}
            \item \textcolor{purple}{Seguro-desemprego}. Pagamento de benefícios a trabalhadores que perdem emprego são socialmente bem-vistos (torna perspectiva de desemprego menos angustiante) mas, se forem suficiente generosos, podem aumentar os salários a uma dada taxa de desemprego\medskip
            \item \textcolor{purple}{Salário mínimo}. Aumentos no salário mínimo podem aumentar não somente o salário mínimo mas, também, todos os salários pouco acima dele, levando a aumento do salário médio, $W$, a uma dada taxa de desemprego\medskip
            \item \textcolor{purple}{Proteções ao emprego}. Um aumento nas proteções ao emprego, que tornam mais caro para empresas a demissão de funcionários, pode aumentar o poder de barganha dos trabalhadores cobertos por essa proteção, aumentando o salário nominal agregado para uma dada taxa de desemprego
        \end{enumerate}
    \end{itemize}
\end{frame}

\begin{frame}
    {Salários, preços e emprego}
    \begin{itemize}
        \item A equação de fixação de salários que vimos anteriormente pode ser reescrita em termos do nível de emprego:
        \begin{equation}
            W = P^e B(N, z)
        \end{equation}
    \end{itemize}
\end{frame}
\section{Bibliografia}
\begin{frame}{\emoji{books} Bibliografia}
    \begin{itemize}                
        \item BLANCHARD, O. Macroeconomia. 7.ed. São Paulo: Pearson Education do Brasil, 2017\medskip                
        \item CARLIN, W.; SOSKICE, D. Macroeconomics: Institutions, instability, and the financial system. Oxford, UK: Oxford University Press, 2015\medskip        
    \end{itemize}
\end{frame}
\end{document}